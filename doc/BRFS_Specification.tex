\documentclass[]{article}
\usepackage[a4paper,margin=1in]{geometry}
\usepackage{fancyhdr}
\usepackage{vhistory}

\pagestyle{fancy}

%title page
\lhead{Working Draft}
\rhead{Revision 0.2b}

\title{
	\textbf{BRFS Specification} \\
	\large Bruno Filesystem (formerly BOOT-ROOT)
}
\author{
	Angel Ruiz Fernandez \textless arf20\textgreater \\
	Bruno Castro García  \textless bruneo32\textgreater
}

\setcounter{tocdepth}{4}
\setcounter{secnumdepth}{4}


\begin{document}

	\maketitle
	\thispagestyle{fancy}
	
	\begin{abstract}
		This specification document describes the BRFS filesystem structure used to store data on storage devices. This provides a standard common description of the filesystem for developers to implement freely.
	\end{abstract}

	\begin{versionhistory}
		\vhEntry{0.1}{}{bruneo32}{Created}
		\vhEntry{0.2}{}{bruneo32}{Unknown}
		\vhEntry{0.2b}{}{bruneo32, arf20}{This document}
	\end{versionhistory}

	\pagebreak
	
	\tableofcontents
	\pagebreak
	
	\section{Introduction}
	\subsection{Scope}
	
	This document defines the Bruno Filesystem. As a filesystem it provides a way of structuring data in a block-based (i.e. LBA) storage device. It is meant for embedded systems where a ciomplex filesystem is not needed, this is not a replacement for any modern desktop filesystem such as ext4, because it lacks basic features of journaling. Although BRFS is able to address large volumes, it is not recommended.
	
	\subsection{Definitions}
	\subsection{Advantages and disadvantages}

\end{document}
